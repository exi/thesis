\chapter{Einführung}\label{ch:einleitung}

Roboter sind aus dem Leben in der modernen industrialisierten Welt kaum noch wegzudenken.
Ob Industrieroboter zur Fertigung von Autos, oder kleiner Serviceroboter in Getränkeautomaten, überall begegnet man mehr oder weniger komplexer Robotik.

Nach wie vor gibt es jedoch einen Bereich, in dem sich die moderne Robotik nach wie vor nicht durchsetzen konnte.
In der (humanoiden) Servicerobotik, deren erklärtes Ziel es ist den Menschen bei seinen alltäglichen Aufgaben so weit wie möglich zu unterstützen, beispielsweise durch das Bereiten von Essen, das Einräumen einer Spülmaschine oder auch in der Altenpflege, existieren noch eine Vielzahl ungelöster, jedoch zentraler Probleme.

Moderne humanoide Roboter wie der am Karlsruher Institut für Technologie entwickelte Serviceroboter ARMAR III sind bereits sehr fortschrittlich mit Hinblick auf Beweglichkeit und Sensorik, jedoch beherrschen sie andere grundlegende Aufgaben wie das Lernen, Abstrahieren, Erkennen und Anwenden von Fakten und Zusammenhängen noch unzuverlässig.

Das Problem der Wiedererkennung bereits wahrgenommener oder ähnlicher Szenen in Innenräumen spielt auf diesem Gebiet eine Herausragende wolle, denn ohne ein korrekte Erkennung einer Situation ist eine Anwendung vorhandenen Wissens nur sehr begrenzt möglich.
Zusätzlich ist ein Verständnis und ein Erlernen einer Szenengeometrie zusammen mit ihren inhärenten Spielräumen und Bewegungen hilfreich für das Einschätzen einer größeren Gesamtsituation.

Das Programmieren durch Vormachen stellt dabei einen intuitiven Ansatz zum Trainieren von Robotern dar.
Erst die Fähigkeit des Lernens aus Demonstration bringt Roboter im Allgemeinen und humanoide Serviceroboter im Speziellen den entscheidenden Schritt in Richtung menschenähnlichen Verhaltens weiter.
Sie wird helfen, den großen Unterschied zwischen beispielsweise einer starren, vorprogrammierten Küchenmaschine und einem echten, auf neue Anforderungen reagierenden Helfer zu überbrücken.

